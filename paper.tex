%%
%% This is file `sample-sigchi.tex',
%% generated with the docstrip utility.
%%
%% The original source files were:
%%
%% samples.dtx  (with options: `sigchi')
%% 
%% IMPORTANT NOTICE:
%% 
%% For the copyright see the source file.
%% 
%% Any modified versions of this file must be renamed
%% with new filenames distinct from sample-sigchi.tex.
%% 
%% For distribution of the original source see the terms
%% for copying and modification in the file samples.dtx.
%% 
%% This generated file may be distributed as long as the
%% original source files, as listed above, are part of the
%% same distribution. (The sources need not necessarily be
%% in the same archive or directory.)
%%
%% The first command in your LaTeX source must be the \documentclass command.
\documentclass[sigchi]{acmart}

%%
%% \BibTeX command to typeset BibTeX logo in the docs
\AtBeginDocument{%
  \providecommand\BibTeX{{%
    \normalfont B\kern-0.5em{\scshape i\kern-0.25em b}\kern-0.8em\TeX}}}

%% Rights management information.  This information is sent to you
%% when you complete the rights form.  These commands have SAMPLE
%% values in them; it is your responsibility as an author to replace
%% the commands and values with those provided to you when you
%% complete the rights form.
\setcopyright{acmcopyright}
\copyrightyear{2019}
\acmYear{2019}
\acmDOI{10.1145/1122445.1122456}

%% These commands are for a PROCEEDINGS abstract or paper.
\acmConference[FPGA '20]{FPGA '20: 28th ACM/SIGDA International Symposium on Field-Programmable Gate Arrays}{February 24--26, 2020}{Seaside, CA}
\acmBooktitle{Proceedings of the 2020 ACM/SIGDA International Symposium on Field-Programmable Gate Arrays}
\acmPrice{15.00}
% Fix this
\acmISBN{978-1-4503-9999-9/18/06}


%%
%% Submission ID.
%% Use this when submitting an article to a sponsored event. You'll
%% receive a unique submission ID from the organizers
%% of the event, and this ID should be used as the parameter to this command.
\acmSubmissionID{123-A56-BU3}

%%
%% The majority of ACM publications use numbered citations and
%% references.  The command \citestyle{authoryear} switches to the
%% "author year" style.
%%
%% If you are preparing content for an event
%% sponsored by ACM SIGGRAPH, you must use the "author year" style of
%% citations and references.
%% Uncommenting
%% the next command will enable that style.
\citestyle{acmauthoryear}

%%
%% end of the preamble, start of the body of the document source.
\begin{document}

%%
%% The "title" command has an optional parameter,
%% allowing the author to define a "short title" to be used in page headers.
\title{Design Space Limitations of Dense, Sparse, Binary, and Ternary
  Neural Networks for Fixed Latency Constraints }

%%
%% The "author" command and its associated commands are used to define
%% the authors and their affiliations.
%% Of note is the shared affiliation of the first two authors, and the
%% "authornote" and "authornotemark" commands
%% used to denote shared contribution to the research.
\author{Double Blind}
\email{doubleblind@doubleblind.com}
%\orcid{0000-0002-5076-7096}
\affiliation{%
  \institution{Double Blind}
  \streetaddress{Double Blind}
  \city{Double Blin}
  \state{DB}
  \country{DB}
  \postcode{92093}
}

%%
%% By default, the full list of authors will be used in the page
%% headers. Often, this list is too long, and will overlap
%% other information printed in the page headers. This command allows
%% the author to define a more concise list
%% of authors' names for this purpose.
\renewcommand{\shortauthors}{DB}

%%
%% The abstract is a short summary of the work to be presented in the
%% article.
\begin{abstract}
Neural network inference on FPGAs has become an effective approach to
perform Machine Learning with large throughput at low
latency. Due to their programmable nature, FPGAs play a distinct role
in the machine learning community  allowing for low latency, low batch,
and low power machine learning inference. Their importantce is particularly
critical for very low latency applications. In this paper, an
exploration of the resource and latency limiations is performed on dense neural networks over a large range of precisions ranging from
binary networks, ternary networks, pruned sparse networks to full
dense networks. As result, we present a rough estimate of the network
sizes that can be allowed for a chosen latency under a fixed set of
FPGA resources. While focus is placed on dense networks, this work is
broadly applicable to all machine learning algorithms.  
\end{abstract}

%%
%% The code below is generated by the tool at http://dl.acm.org/ccs.cfm.
%% Please copy and paste the code instead of the example below.
%%


\begin{CCSXML}
<ccs2012>
<concept>
<concept_id>10010147.10010257</concept_id>
<concept_desc>Computing methodologies~Machine learning</concept_desc>
<concept_significance>500</concept_significance>
</concept>
<concept>
<concept_id>10010147.10010257.10010293.10010294</concept_id>
<concept_desc>Computing methodologies~Neural networks</concept_desc>
<concept_significance>500</concept_significance>
</concept>
<concept>
<concept_id>10010520.10010553.10010562</concept_id>
<concept_desc>Computer systems organization~Embedded systems</concept_desc>
<concept_significance>500</concept_significance>
</concept>
<concept>
<concept_id>10010520.10010553.10010562.10010561</concept_id>
<concept_desc>Computer systems organization~Firmware</concept_desc>
<concept_significance>500</concept_significance>
</concept>
<concept>
<concept_id>10010520.10010570</concept_id>
<concept_desc>Computer systems organization~Real-time systems</concept_desc>
<concept_significance>500</concept_significance>
</concept>
</ccs2012>
\end{CCSXML}

\ccsdesc[500]{Computing methodologies~Machine learning}
\ccsdesc[500]{Computing methodologies~Neural networks}
\ccsdesc[500]{Computer systems organization~Embedded systems}
\ccsdesc[500]{Computer systems organization~Firmware}
\ccsdesc[500]{Computer systems organization~Real-time systems}


%%
%% Keywords. The author(s) should pick words that accurately describe
%% the work being presented. Separate the keywords with commas.
\keywords{graph neural networks, FPGA}


%%
%% This command processes the author and affiliation and title
%% information and builds the first part of the formatted document.
\maketitle

\section{Introduction}
Due to their programmable nature, implementations of machine learning algorithms on FPGAs have a
unique set of properties. In particular, algorithms can be partly or
fully unrolled onto the processor allowing for variable latencies and an optimized use of
resources. 

\section{Related Work}
Related work.

\section{Implementation}
Implementation.

\section{Performance}
Performance.

\section{Summary}
Summary.

%%
%% The acknowledgments section is defined using the "acks" environment
%% (and NOT an unnumbered section). This ensures the proper
%% identification of the section in the article metadata, and the
%% consistent spelling of the heading.
\begin{acks}
Acknowledgements.
\end{acks}

%%
%% The next two lines define the bibliography style to be used, and
%% the bibliography file.
\bibliographystyle{ACM-Reference-Format}
\bibliography{sample-base}

%%
\appendix

\section{Appendix}

Appendix.

\end{document}
\endinput
%%
%% End of file `sample-sigchi.tex'.
